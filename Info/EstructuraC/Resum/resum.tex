\documentclass[a4paper,10pt]{article}
\usepackage[utf8]{inputenc}
\usepackage[catalan]{babel}

% Opino que la vida sense color no es vida
\usepackage{color}
\newcommand{\blue}[1]{{\color{blue}#1}}
\newcommand{\red}[1]{{\color{red}#1}}

\begin{document}
\tableofcontents\newpage

\part{\red{D}efinicions}
\begin{itemize}
\item 1 Nibble = 4 bits
\item[K] $\to 10^3, 2^{10}$
\item[M] $\to 10^6, 2^{20}$
\item[G] $\to 10^9, 2^{30}$
\item Prestacions
	\begin{itemize}
	\item Freqüència \red{Hz}
	\item Transistors
	\item Memòria principal
		\subitem Grandària = \red{n} posicions $\times$ \red{n} Grandària de la posició
	\item Amplada de Banda (bus)
		\subitem \blue{Freqüència $\times$ \red{n} bits del bus}
	\item Programes prova
		\subitem És aquí que es descobreix quin va millor realment
	\end{itemize}
\item Cicle de Rellotge
	\subitem \blue{Quan tarda el rellotge en fer tot un cicle}
\item Cicle d'instrucció
	\subitem \blue{Quan tarda en fer la instrucció demanada}
\item Cicle de bus
	\subitem \blue{Temps que tarda en enviar dades}
\end{itemize}

\part{Parts d'un ordinador}
\section{Bàsic}
\begin{itemize}
\item CPU
	\begin{itemize}
	\item UC \blue{-unitat de control-}
	\item UP \blue{-unitat d'execució-}
	\end{itemize}
\item Memòria Principal
\item Dispositius d'E/S
\end{itemize}

\part{\red{E}volució}
\section{Microprocessadors}
\begin{itemize}
\item[1er] 4004
	\begin{itemize}
	\item Bus 4 bits
	\item Direcció 12 bits
	\end{itemize}
\item 1971 - 2011
	\begin{itemize}
	\item $10^6$ Transistors \blue{2.300 - 2.270.000.000}
	\item $10^7$ Freqüència \blue{100 kHz - 4 GHz}
	\end{itemize}
\item Llei de Moore -1.965-
	\subitem 24 mesos $\to$  2 $\times$ transistors
\item Amortir els dissenys
	\begin{itemize}
	\item High Performance
	\item Low Power
	\item Traditional Embedded
	\end{itemize}
\end{itemize}

\section{Factor tecnològic}
\begin{itemize}
\item Transistor \blue{vàlvules i} \red{relets}
\end{itemize}
\end{document}

Vaig per el EC 01 - 2
coses a comprovar diu el profe José Bosch: 1024 x 1024 = 2^10... rar de collons
