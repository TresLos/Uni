\documentclass[a4paper,10pt]{article}
\usepackage[utf8]{inputenc}
\usepackage[catalan]{babel}

% Opino que la vida sense color no es vida
\usepackage{color}
\newcommand{\blue}[1]{{\color{blue}#1}}
\newcommand{\red}[1]{{\color{red}#1}}


\usepackage{amssymb, amsmath}
\newcommand{\A}{\mathbb{A}}

\begin{document}
\tableofcontents\newpage
\part{Geometria de Euclides segle IIIaC}
\begin{enumerate}
\item 2 punts del pla
	\subitem $p \neq q$
	\subitem $\exists !$ recta que passa per $p$ i $q$
\item una recta
	\subitem Pot créixer indefinidament per cada segment
\item \phantom{Doncs no ho ha dit res}
\item $\forall \angle \perp$ is $\equiv$
\item $\forall p \notin r, \quad \exists ! s: p \in s \wedge r \parallel s$
\end{enumerate}

\part{\red{D}efinició}
\section{Espais afins}
% ultra elemental, és trist fer-ho
\begin{itemize}
\item Espai vectorial
	\begin{itemize}
	\item \blue{sobre el} cos $K$ \red{escalars}
	\item \blue{és un} conjunt $E$ \red{vectors}
	\item Operacions internes $(+, \times)$
		\begin{itemize}
		\item[$+$:] $$	\begin{array}{rcl}
				E\times E		& \to		& E			\\
				(\vec{u}, \vec{v})	& \mapsto	& \vec{u} + \vec{v}
				\end{array}$$
		\item[$\times$:] $$	\begin{array}{rcl}
				K\times E		& \to		& E			\\
				(\lambda, \vec{u})	& \mapsto	& \lambda \vec{u}
				\end{array}$$
		\end{itemize}
	\end{itemize}
\item Espai afí
	\begin{itemize}
	\item $\A \times E$ \red{vectors fixos}
	\item \blue{Associat} Espai vectorial $E$ \red{vectors lliures}
	\item \blue{Al} cos $K$
	\item \blue{És un} conjunt $\A$ \red{punts}
	\item Operació interna $(+)$
		\begin{itemize}
		\item[$\phi:$] $$\begin{array}{rcl}
				\A \times E	& \to		& \A	\\
				(p, \vec{u})	& \mapsto	& p + \vec{u}
				\end{array}$$
		\end{itemize}
	\item Axiomes
		\begin{enumerate}
		\item $$\begin{array}{rcl}E&\to&\A\\\vec{u}&\mapsto&p+\vec{u}\end{array}$$ \blue{és} bijectiva
		\item $\forall p \in \A \quad \vec{u}, \vec{v} \in E$
			\subitem $(p + \vec{u}) + \vec{v} = p + (\vec{u} + \vec{v})$
			\subitem $\phi (\phi (p, \vec{u}), \vec{v}) = \phi (p, \vec{u} + \vec{v})$
		\item $\dim{\A} = \dim{E}$
			\subitem $\phi: \begin{array}{rcl}\A \times E&\to&\A\\(p,\vec{u})&\mapsto&p+\vec{u}\end{array}$
			\subitem \blue{la suma de} $E \forall$ espai vectorial \blue{és} afí
		\item $(\A, E, \phi)$ \blue{és un} espai afí
			\subitem $\dim{\A} = \dim{E}$
		\end{enumerate}
	\end{itemize}
	\item Proposició
		\begin{itemize}
		\item Els axiomes 1 i 2 són equivalents
		\item $\forall p, q \in \A\quad \exists: \vec{u} \in E:\quad p + \vec{u} = q = p + \overrightarrow{pq}$
		\end{itemize}
	\item Demostració
		\begin{itemize}
		\item 1, bijectiva
		\item 2
			\begin{itemize}
			\item $r = p + \overrightarrow{pr}$
			\item $q = p + \overrightarrow{pq}$
				\subitem $\overrightarrow{pq} = \overrightarrow{pr} + \overrightarrow{rq}$
			\end{itemize}
		\end{itemize}
	\item Coro\lgem ari % ta ben escrit
		\begin{enumerate}
		\item $\overrightarrow{pp} = 0$
		\item $p + \vec{0} = p$
		\item $\overrightarrow{pq} = \vec{0} \Rightarrow p = q$
		\end{enumerate}
\end{itemize}

\part{Ara em fa pal ordenar. Espero que ja ho faré}
\section{2 - Combinacions ``lineals'' de punts}
\subsection*{Proposició}
$$\sum \lambda_i p_i\quad \text{si } \sum \lambda_i = 1$$
\subsection*{Demostració}
$$\forall 0, 0' \in \A \quad 0 + \left(\sum \lambda_i\right)\overrightarrow{00'} = 0'$$

\section{3 - Baricentre}
$$B:= \sum^k_{i=1} \frac{1}{k} p_i$$
\subsection*{Proposició}
$B$ és l'únic punt de $\A$ que verifica:
$$\sum \overrightarrow{Bp_i} = 0$$
%\subsection*{Demostració}, hi ha una, però ara mateix la desconec

\end{document}
